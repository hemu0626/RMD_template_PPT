% 中文
%\usepackage{fontspec, xunicode, xltxtra}    
\usepackage{ctex}%中文字体
%\setCJKfamilyfont{song}{SimSun} 
%\setCJKfamilyfont{hei}{SimHei}
\usepackage{tikz}
\usepackage{pdfrender}

%\usepackage[orientation=landscape,size=custom,width=16,height=9,scale=0.5,debug]{beamerposter} 


% 宏包
%---------------------
\usepackage{setspace}
\usepackage{xcolor}
\usepackage{graphicx} % 插入图片
%\usepackage[english]{babel} % 新版本 CTEX2.9.X必须
%
\usepackage{marvosym}

\usepackage{booktabs} % Allows the use of \toprule, \midrule and \bottomrule in tables
\usepackage{animate}
%\usepackage{hyperref}
\hypersetup{
	unicode={true},
	bookmarksopen={true},
	pdfborder={0 0 0},
	citecolor=blue,
	linkcolor=blue, 
	anchorcolor=blue,
	urlcolor=blue,
	colorlinks=true,     
	pdfborder=000       
}

% 设定图表caption
\usepackage{caption}
\captionsetup{%
	figurename=图,
	tablename=表
}

\setbeamertemplate{theorems}[numbered]
\setbeamertemplate{caption}[numbered]


% 定义一些自选的模板,包括背景、图标、导航条和页脚等,修改要慎重
% 设置背景渐变由10%的红变成10%的结构颜色
%\beamertemplateshadingbackground{red!10}{structure!10}
% %\beamertemplatesolidbackgroundcolor{white!90!blue}
% 使所有隐藏的文本完全透明、动态,而且动态的范围很小
\beamertemplatetransparentcovereddynamic
% 使itemize环境中变成小球,这是一种视觉效果
\beamertemplateballitem
% 为所有已编号的部分设置一个章节目录,并且编号显示成小球
\beamertemplatenumberedballsectiontoc
% 将每一页的要素的要素名设成加粗字体
\beamertemplateboldpartpage
% item逐步显示时,使已经出现的item、正在显示的item、将要出现的item呈现不同颜色
\def\hilite<#1>{
	\temporal<#1>{\color{gray}}{\color{blue}}
	{\color{blue!25}}
}
% 自定义彩色块状结构的颜色
\setbeamercolor{bgcolor}{fg=yellow,bg=cyan}

% 手形item, 需要marvosym
\setbeamertemplate{itemize item}{\color{red}\Large\Pointinghand}
\setbeamertemplate{itemize subitem}{\color{red}\Writinghand}


\usepackage{textpos}
\addtobeamertemplate{frametitle}{}{%
	\begin{textblock*}{100mm}(.94\textwidth,-0.96cm)
		\includegraphics[height=0.96cm,width=2.0cm,keepaspectratio=TRUE]{statlogo-02.png}
	\end{textblock*}}
	
	%%%%%%%%%%%%%%%%%%%%%%%%%%%%%%%%%%%%%%%%%%%%%%%
	
	\graphicspath{{figures/}}
	
	\everydisplay{\color{red}}
	\setbeamercovered{transparent}
	%\beamerdefaultoverlayspecification{<+->}
	
	\AtBeginSection[]
	{
		\begin{frame}
			\frametitle{报告提纲}
			\tableofcontents[currentsection,hideallsubsections]
		\end{frame}
	}
	\AtBeginSubsection[]
	{
		\begin{frame}[shrink]
			\frametitle{报告提纲}
			\begin{spacing}{1.4}
				\tableofcontents[sectionstyle=show/shaded,subsectionstyle=show/shaded/hide]
			\end{spacing}
		\end{frame}
	}
	
	\usetikzlibrary{shapes,arrows}
	\setbeamerfont{author}{size=\Huge}
	\setbeamerfont{institute}{size=\normalsize\itshape}
	\setbeamerfont{title}{size=\fontsize{30}{36}\bfseries}
	\setbeamerfont{subtitle}{size=\Large\normalfont\slshape}
	
	% Define title page style 
	\setbeamertemplate{title page}{%
		\begin{tikzpicture}[remember picture,overlay]
		\fill[orange]
		([yshift=25mm]current page.west) rectangle (current page.south east);
		\node[anchor=east]
		at ([yshift=30pt,xshift=-20pt]current page.east) (title)
		{\parbox[t]{\textwidth}{\raggedleft%
				\usebeamerfont{author}\textcolor{white}{%
					\textpdfrender{
						TextRenderingMode=FillStroke,
						FillColor=white,
						LineWidth=.1ex,
					}{\inserttitle}}}};
		\node[anchor=east]
		at ([yshift=-35pt,xshift=-20pt]current page.north east) (subtitle)
		{\parbox[t]{.6\paperwidth}{\raggedleft%
				\usebeamerfont{subtitle}\textcolor{blue}{\insertsubtitle}}};
		\node[anchor=east]
		at ([yshift=-50pt,xshift=-20pt]current page.east) (institute)
		{\parbox[t]{.78\paperwidth}{\raggedleft%
				\usebeamerfont{institute}\textcolor{blue}{\insertinstitute}}};
		\node[anchor=east]
		at ([yshift=-10pt,xshift=-20pt]current page.east) (author)
		{\parbox[t]{.6\paperwidth}{\raggedleft%
				\usebeamerfont{author}\textcolor{blue}{%
					\textpdfrender{
						TextRenderingMode=FillStroke,
						FillColor=orange,
						LineWidth=.1ex,
					}{\insertauthor}}}};
		\node[anchor=north west]
		at ([yshift=2pt,xshift=-10.0pt]current page.north west) (logo)
		{\parbox[t]{.19\paperwidth}{\raggedleft%
				\usebeamercolor[fg]{titlegraphic}\inserttitlegraphic}};
		\end{tikzpicture}
	}
	
	\usebackgroundtemplate{% 
		\tikz\node[opacity=0.3,inner sep=0] {\includegraphics[height=\paperheight,width=\paperwidth,keepaspectratio=FALSE]{bdbg-2.jpg}};} % bdbg-1.jpeg
	
	%\titlegraphic{\includegraphics[width=3.0cm]{Shvad-Chi.png}}
	\titlegraphic{\includegraphics[width=2.2cm]{statlogo-01.png}} %stat-right.png
